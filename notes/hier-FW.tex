\section{Hierarchical Floyd Warshall}

Let the navigation problem be defined as finding a sequence of goals in an unexplored maze.
The goal image is provided in terms of observations.
When the agent reaches the goal, it is rewarded.

Let the policy be a function of hierarchical states
$\policy(\act | \state_1, \state_2, \state_{g1}, \state_{g2})$.
The higher level policy only depends on high level state and high level goals and yields lower level policy.
$\policy_2( \policy_1 | \state_2, \state_{g2} )$.

How can CNNs model hierarchical functions?



CNNs find repeated local patterns in data.
A time dimensional CNN can also find local patterns in state space,
which will hopefully capture repetition of certain aspect of the states.
That does not guarantee hierarchy though.

Sparsity and hierarchy are related and are hard to formalize?

Define sparsity of state vector?

\begin{definition}
  A MDP $(\State, \Action, \Trans, \R)$ is said to be sparsely
  connected if $\Trans$ has a block diagonal structure and the number
  of non-zero entities is much less than zero entities.
\end{definition}

\begin{definition}
  A MDP $(\State, \Action, \Trans, \R)$ is said to be hierarchical if
  $\Trans$ and $\R$ have repeated patterns and the transition matrix
  can be factorized into $\Trans = \Trans_1 \odot_k \Trans_2$, where
  $\odot_k$ is the kronecker product.
  In terms of functions $\Trans(\state_1, \state_2) = \Trans_1(\state_1)\Trans_2(\state_2)$.
  Similarly the reward function should also be factorizable.
  In terms of functions $\rew(\state_1, \state_2) = \rew_1(\state_1) + \rew_2(\state_2)$.
\end{definition}
